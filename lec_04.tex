\lecture{3}{17/03/2023}{}

\subsection{5.6}
For an integral that behaves like:
\begin{align*}
    \int^{\Lambda}q^{\omega(G)-1}\dd q
\end{align*}
We can define the superficial degree of divergence for an integral over q %TODO
\begin{equation*}
    \omega(G)=LD-2I
\end{equation*}
The integral is divergent for $\omega(G)>\geq0$ and superficially convergent for $\omega(G)<0$, meaning that integration over a subset of the domain may not converge.
It is possible to derive an expression for this superficial degree in terms of the number of external lines of the diagram.
\\

\subsubsection{Topological argument}
Consider an interaction permitting various powers of $\phi$ and its derivatives eg. $\phi(\nabla\phi)^2$.
From Wick's theorem:
\begin{align*}
    \overline{\phi(x)\nabla\phi(z)}\rightarrow\nabla_z G_0(x-z)=F[ikG_0(k)]
\end{align*}
Where $F[g]$ denotes the Fourier transform of $g$. Thus the derivative has the effect of multiplying the the vertex by a factor of $ik$.

